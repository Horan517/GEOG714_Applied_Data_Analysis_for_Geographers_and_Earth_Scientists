% Options for packages loaded elsewhere
\PassOptionsToPackage{unicode}{hyperref}
\PassOptionsToPackage{hyphens}{url}
%
\documentclass[
]{article}
\usepackage{amsmath,amssymb}
\usepackage{iftex}
\ifPDFTeX
  \usepackage[T1]{fontenc}
  \usepackage[utf8]{inputenc}
  \usepackage{textcomp} % provide euro and other symbols
\else % if luatex or xetex
  \usepackage{unicode-math} % this also loads fontspec
  \defaultfontfeatures{Scale=MatchLowercase}
  \defaultfontfeatures[\rmfamily]{Ligatures=TeX,Scale=1}
\fi
\usepackage{lmodern}
\ifPDFTeX\else
  % xetex/luatex font selection
\fi
% Use upquote if available, for straight quotes in verbatim environments
\IfFileExists{upquote.sty}{\usepackage{upquote}}{}
\IfFileExists{microtype.sty}{% use microtype if available
  \usepackage[]{microtype}
  \UseMicrotypeSet[protrusion]{basicmath} % disable protrusion for tt fonts
}{}
\makeatletter
\@ifundefined{KOMAClassName}{% if non-KOMA class
  \IfFileExists{parskip.sty}{%
    \usepackage{parskip}
  }{% else
    \setlength{\parindent}{0pt}
    \setlength{\parskip}{6pt plus 2pt minus 1pt}}
}{% if KOMA class
  \KOMAoptions{parskip=half}}
\makeatother
\usepackage{xcolor}
\usepackage[margin=1in]{geometry}
\usepackage{color}
\usepackage{fancyvrb}
\newcommand{\VerbBar}{|}
\newcommand{\VERB}{\Verb[commandchars=\\\{\}]}
\DefineVerbatimEnvironment{Highlighting}{Verbatim}{commandchars=\\\{\}}
% Add ',fontsize=\small' for more characters per line
\usepackage{framed}
\definecolor{shadecolor}{RGB}{248,248,248}
\newenvironment{Shaded}{\begin{snugshade}}{\end{snugshade}}
\newcommand{\AlertTok}[1]{\textcolor[rgb]{0.94,0.16,0.16}{#1}}
\newcommand{\AnnotationTok}[1]{\textcolor[rgb]{0.56,0.35,0.01}{\textbf{\textit{#1}}}}
\newcommand{\AttributeTok}[1]{\textcolor[rgb]{0.13,0.29,0.53}{#1}}
\newcommand{\BaseNTok}[1]{\textcolor[rgb]{0.00,0.00,0.81}{#1}}
\newcommand{\BuiltInTok}[1]{#1}
\newcommand{\CharTok}[1]{\textcolor[rgb]{0.31,0.60,0.02}{#1}}
\newcommand{\CommentTok}[1]{\textcolor[rgb]{0.56,0.35,0.01}{\textit{#1}}}
\newcommand{\CommentVarTok}[1]{\textcolor[rgb]{0.56,0.35,0.01}{\textbf{\textit{#1}}}}
\newcommand{\ConstantTok}[1]{\textcolor[rgb]{0.56,0.35,0.01}{#1}}
\newcommand{\ControlFlowTok}[1]{\textcolor[rgb]{0.13,0.29,0.53}{\textbf{#1}}}
\newcommand{\DataTypeTok}[1]{\textcolor[rgb]{0.13,0.29,0.53}{#1}}
\newcommand{\DecValTok}[1]{\textcolor[rgb]{0.00,0.00,0.81}{#1}}
\newcommand{\DocumentationTok}[1]{\textcolor[rgb]{0.56,0.35,0.01}{\textbf{\textit{#1}}}}
\newcommand{\ErrorTok}[1]{\textcolor[rgb]{0.64,0.00,0.00}{\textbf{#1}}}
\newcommand{\ExtensionTok}[1]{#1}
\newcommand{\FloatTok}[1]{\textcolor[rgb]{0.00,0.00,0.81}{#1}}
\newcommand{\FunctionTok}[1]{\textcolor[rgb]{0.13,0.29,0.53}{\textbf{#1}}}
\newcommand{\ImportTok}[1]{#1}
\newcommand{\InformationTok}[1]{\textcolor[rgb]{0.56,0.35,0.01}{\textbf{\textit{#1}}}}
\newcommand{\KeywordTok}[1]{\textcolor[rgb]{0.13,0.29,0.53}{\textbf{#1}}}
\newcommand{\NormalTok}[1]{#1}
\newcommand{\OperatorTok}[1]{\textcolor[rgb]{0.81,0.36,0.00}{\textbf{#1}}}
\newcommand{\OtherTok}[1]{\textcolor[rgb]{0.56,0.35,0.01}{#1}}
\newcommand{\PreprocessorTok}[1]{\textcolor[rgb]{0.56,0.35,0.01}{\textit{#1}}}
\newcommand{\RegionMarkerTok}[1]{#1}
\newcommand{\SpecialCharTok}[1]{\textcolor[rgb]{0.81,0.36,0.00}{\textbf{#1}}}
\newcommand{\SpecialStringTok}[1]{\textcolor[rgb]{0.31,0.60,0.02}{#1}}
\newcommand{\StringTok}[1]{\textcolor[rgb]{0.31,0.60,0.02}{#1}}
\newcommand{\VariableTok}[1]{\textcolor[rgb]{0.00,0.00,0.00}{#1}}
\newcommand{\VerbatimStringTok}[1]{\textcolor[rgb]{0.31,0.60,0.02}{#1}}
\newcommand{\WarningTok}[1]{\textcolor[rgb]{0.56,0.35,0.01}{\textbf{\textit{#1}}}}
\usepackage{graphicx}
\makeatletter
\def\maxwidth{\ifdim\Gin@nat@width>\linewidth\linewidth\else\Gin@nat@width\fi}
\def\maxheight{\ifdim\Gin@nat@height>\textheight\textheight\else\Gin@nat@height\fi}
\makeatother
% Scale images if necessary, so that they will not overflow the page
% margins by default, and it is still possible to overwrite the defaults
% using explicit options in \includegraphics[width, height, ...]{}
\setkeys{Gin}{width=\maxwidth,height=\maxheight,keepaspectratio}
% Set default figure placement to htbp
\makeatletter
\def\fps@figure{htbp}
\makeatother
\setlength{\emergencystretch}{3em} % prevent overfull lines
\providecommand{\tightlist}{%
  \setlength{\itemsep}{0pt}\setlength{\parskip}{0pt}}
\setcounter{secnumdepth}{-\maxdimen} % remove section numbering
\usepackage{listings}
\usepackage{xcolor}
\lstset{
  breaklines=true,        % 自动换行
  breakatwhitespace=true, % 仅在空白处换行,确保注释保持整洁
  basicstyle=\ttfamily,   % 使用等宽字体
  keepspaces=true,        % 保留空格
  columns=flexible,       % 灵活的列宽
  showstringspaces=false, % 不显示字符串中的空格
  escapeinside={(*@}{@*)}, % 用于在代码中插入LaTeX命令
  commentstyle=\color{gray},  % 注释的颜色
  xleftmargin=1em,        % 左侧边距
  xrightmargin=1em        % 右侧边距
\ifLuaTeX
  \usepackage{selnolig}  % disable illegal ligatures
\fi
\usepackage{bookmark}
\IfFileExists{xurl.sty}{\usepackage{xurl}}{} % add URL line breaks if available
\urlstyle{same}
\hypersetup{
  pdftitle={GEOG 714 - Assignment 3},
  pdfauthor={Haoran Xu},
  hidelinks,
  pdfcreator={LaTeX via pandoc}}

\title{GEOG 714 - Assignment 3}
\author{Haoran Xu}
\date{Sep 29, 2024}

\begin{document}
\maketitle

\begin{Shaded}
\begin{Highlighting}[]
\CommentTok{\# hi!  don\textquotesingle{}t know why this can\textquotesingle{}t be detected!}

\NormalTok{a }\OtherTok{\textless{}{-}} \DecValTok{5}
\NormalTok{change }\OtherTok{\textless{}{-}} \ControlFlowTok{function}\NormalTok{(x) \{}
    \FunctionTok{return}\NormalTok{(x }\SpecialCharTok{+}\NormalTok{ a)}
\NormalTok{\}}
\FunctionTok{change}\NormalTok{(}\DecValTok{5}\NormalTok{)}
\CommentTok{\# print(x) \# this will generate an error}
\end{Highlighting}
\end{Shaded}

\begin{Shaded}
\begin{Highlighting}[]
\NormalTok{a }\OtherTok{\textless{}{-}} \DecValTok{5}
\NormalTok{x }\OtherTok{\textless{}{-}} \DecValTok{6}
\NormalTok{change }\OtherTok{\textless{}{-}} \ControlFlowTok{function}\NormalTok{(x) \{}
\NormalTok{    y }\OtherTok{\textless{}{-}} \DecValTok{5}
    \FunctionTok{return}\NormalTok{(x }\SpecialCharTok{+}\NormalTok{ a }\SpecialCharTok{+}\NormalTok{ y)}
\NormalTok{\}}
\FunctionTok{change}\NormalTok{(}\DecValTok{5}\NormalTok{)}
\FunctionTok{print}\NormalTok{(x)}
\CommentTok{\# print(y) \# this will generate an error}
\end{Highlighting}
\end{Shaded}

\begin{Shaded}
\begin{Highlighting}[]
\NormalTok{stuff }\OtherTok{\textless{}{-}} \DecValTok{1}
\NormalTok{blah }\OtherTok{\textless{}{-}} \ControlFlowTok{function}\NormalTok{() \{}
\NormalTok{    stuff }\OtherTok{\textless{}{-}} \DecValTok{5}
\NormalTok{\}}
\FunctionTok{print}\NormalTok{(stuff)}
\FunctionTok{blah}\NormalTok{()}
\FunctionTok{print}\NormalTok{(stuff)}

\CommentTok{\# compare below:}
\NormalTok{stuff }\OtherTok{\textless{}{-}} \DecValTok{1}
\NormalTok{blah }\OtherTok{\textless{}{-}} \ControlFlowTok{function}\NormalTok{() \{}
\NormalTok{    stuff }\OtherTok{\textless{}\textless{}{-}} \DecValTok{5}  \CommentTok{\# \textquotesingle{}\textless{}\textless{}{-}\textquotesingle{} assignment operator can be used to change global variables from within functions. }
\NormalTok{\}}
\FunctionTok{print}\NormalTok{(stuff)}
\FunctionTok{blah}\NormalTok{()}
\FunctionTok{print}\NormalTok{(stuff)}
\end{Highlighting}
\end{Shaded}

\begin{Shaded}
\begin{Highlighting}[]
\CommentTok{\# install.packages(\textquotesingle{}here\textquotesingle{})}
\FunctionTok{library}\NormalTok{(here)}
\end{Highlighting}
\end{Shaded}

\begin{verbatim}
## here() starts at /Users/horanxu/Desktop/GEOG714_Applied_Data_Analysis_for_Geographers_and_Earth_Scientists
\end{verbatim}

\begin{Shaded}
\begin{Highlighting}[]
\FunctionTok{here}\NormalTok{()}

\NormalTok{df }\OtherTok{\textless{}{-}} \FunctionTok{as.data.frame}\NormalTok{(}\FunctionTok{read.csv}\NormalTok{(}\FunctionTok{paste0}\NormalTok{(}\FunctionTok{here}\NormalTok{(), }\StringTok{"/Assignments/Assignment3/data/Canada2006\_WVS\_Sheet1.csv"}\NormalTok{)))}

\FunctionTok{names}\NormalTok{(df) }\OtherTok{\textless{}{-}} \FunctionTok{c}\NormalTok{(}\StringTok{"respondent"}\NormalTok{, }\StringTok{"happiness"}\NormalTok{, }\StringTok{"healthg"}\NormalTok{, }\StringTok{"friends"}\NormalTok{, }\StringTok{"satisfaction"}\NormalTok{, }\StringTok{"membership"}\NormalTok{,}
    \StringTok{"science"}\NormalTok{, }\StringTok{"age"}\NormalTok{, }\StringTok{"size"}\NormalTok{, }\StringTok{"weight"}\NormalTok{)}

\NormalTok{m }\OtherTok{\textless{}{-}} \FunctionTok{matrix}\NormalTok{(}\FunctionTok{c}\NormalTok{(}\DecValTok{1}\NormalTok{, }\DecValTok{4}\NormalTok{, }\DecValTok{3}\NormalTok{, }\DecValTok{2}\NormalTok{, }\SpecialCharTok{{-}}\DecValTok{1}\NormalTok{, }\DecValTok{5}\NormalTok{, }\DecValTok{3}\NormalTok{, }\DecValTok{4}\NormalTok{, }\SpecialCharTok{{-}}\DecValTok{1}\NormalTok{), }\AttributeTok{ncol =} \DecValTok{3}\NormalTok{, }\AttributeTok{nrow =} \DecValTok{3}\NormalTok{)}
\FunctionTok{apply}\NormalTok{(m, }\DecValTok{1}\NormalTok{, }\ControlFlowTok{function}\NormalTok{(x) x }\SpecialCharTok{\textless{}} \DecValTok{0}\NormalTok{)  }\CommentTok{\# return TRUE/FALSE}
\NormalTok{m[}\FunctionTok{apply}\NormalTok{(m, }\DecValTok{1}\NormalTok{, }\ControlFlowTok{function}\NormalTok{(x) x }\SpecialCharTok{\textless{}} \DecValTok{0}\NormalTok{)] }\OtherTok{\textless{}{-}} \DecValTok{0}
\end{Highlighting}
\end{Shaded}

\begin{Shaded}
\begin{Highlighting}[]
\NormalTok{df[df }\SpecialCharTok{\textless{}} \DecValTok{0}\NormalTok{] }\OtherTok{\textless{}{-}} \ConstantTok{NA}  \CommentTok{\# no need add quotation marks}
\FunctionTok{length}\NormalTok{(df}\SpecialCharTok{$}\NormalTok{healthg[}\FunctionTok{is.na}\NormalTok{(df}\SpecialCharTok{$}\NormalTok{healthg) }\SpecialCharTok{==} \ConstantTok{TRUE}\NormalTok{])}
\end{Highlighting}
\end{Shaded}

\subsection{Q1. Write a two to three sentence explanation about what is
happening in this code. You may need to read up on the length() and
is.na()
functions.}\label{q1.-write-a-two-to-three-sentence-explanation-about-what-is-happening-in-this-code.-you-may-need-to-read-up-on-the-length-and-is.na-functions.}

The code uses \texttt{is.na()} to identify missing values in the
\texttt{healthg} column and applies \texttt{length()} to count the total
number of \texttt{NA} entries in that column. So it can calcutes the
number of missing values of \texttt{df\$healthg}.

\begin{Shaded}
\begin{Highlighting}[]
\NormalTok{df }\OtherTok{\textless{}{-}}\NormalTok{ df[}\FunctionTok{apply}\NormalTok{(df, }\DecValTok{1}\NormalTok{, }\ControlFlowTok{function}\NormalTok{(x) }\SpecialCharTok{!}\FunctionTok{any}\NormalTok{(}\FunctionTok{is.na}\NormalTok{(x))), ]  }\CommentTok{\# return TRUE/FALSE}
\CommentTok{\# The innermost function is.na() returns TRUE if a value is NA.  The any()}
\CommentTok{\# function returns TRUE if any value inside the row is TRUE.  The ‘!’ operator}
\CommentTok{\# reverses any() so that it is FALSE if any value in the row is TRUE.  If any}
\CommentTok{\# value in any row is NA, then that row is not included in the output file.}

\NormalTok{df}\SpecialCharTok{$}\NormalTok{happiness }\OtherTok{\textless{}{-}} \FunctionTok{as.factor}\NormalTok{(df}\SpecialCharTok{$}\NormalTok{happiness)}
\end{Highlighting}
\end{Shaded}

\subsection{Q2. Write code to convert all of these variables (healthg,
friends, satisfaction, membership, science,size) into factor
variables}\label{q2.-write-code-to-convert-all-of-these-variables-healthg-friends-satisfaction-membership-sciencesize-into-factor-variables}

\begin{Shaded}
\begin{Highlighting}[]
\NormalTok{df}\SpecialCharTok{$}\NormalTok{healthg }\OtherTok{\textless{}{-}} \FunctionTok{as.factor}\NormalTok{(df}\SpecialCharTok{$}\NormalTok{healthg)}
\NormalTok{df}\SpecialCharTok{$}\NormalTok{friends }\OtherTok{\textless{}{-}} \FunctionTok{as.factor}\NormalTok{(df}\SpecialCharTok{$}\NormalTok{friends)}
\NormalTok{df}\SpecialCharTok{$}\NormalTok{satisfaction }\OtherTok{\textless{}{-}} \FunctionTok{as.factor}\NormalTok{(df}\SpecialCharTok{$}\NormalTok{satisfaction)}
\NormalTok{df}\SpecialCharTok{$}\NormalTok{membership }\OtherTok{\textless{}{-}} \FunctionTok{as.factor}\NormalTok{(df}\SpecialCharTok{$}\NormalTok{membership)}
\NormalTok{df}\SpecialCharTok{$}\NormalTok{science }\OtherTok{\textless{}{-}} \FunctionTok{as.factor}\NormalTok{(df}\SpecialCharTok{$}\NormalTok{science)}
\NormalTok{df}\SpecialCharTok{$}\NormalTok{size }\OtherTok{\textless{}{-}} \FunctionTok{as.factor}\NormalTok{(df}\SpecialCharTok{$}\NormalTok{size)}
\end{Highlighting}
\end{Shaded}

\begin{Shaded}
\begin{Highlighting}[]
\FunctionTok{unique}\NormalTok{(df}\SpecialCharTok{$}\NormalTok{happiness)  }\CommentTok{\# used in factors}
\FunctionTok{unique}\NormalTok{(df}\SpecialCharTok{$}\NormalTok{healthg)}

\NormalTok{df}\SpecialCharTok{$}\NormalTok{happiness }\OtherTok{\textless{}{-}} \FunctionTok{as.character}\NormalTok{(df}\SpecialCharTok{$}\NormalTok{happiness)}
\NormalTok{df}\SpecialCharTok{$}\NormalTok{healthg }\OtherTok{\textless{}{-}} \FunctionTok{as.character}\NormalTok{(df}\SpecialCharTok{$}\NormalTok{healthg)}

\NormalTok{df}\SpecialCharTok{$}\NormalTok{happiness[df}\SpecialCharTok{$}\NormalTok{happiness }\SpecialCharTok{==} \StringTok{"1"}\NormalTok{] }\OtherTok{\textless{}{-}} \StringTok{"1. Very happy"}
\NormalTok{df}\SpecialCharTok{$}\NormalTok{happiness[df}\SpecialCharTok{$}\NormalTok{happiness }\SpecialCharTok{==} \StringTok{"2"}\NormalTok{] }\OtherTok{\textless{}{-}} \StringTok{"2. Quite happy"}
\NormalTok{df}\SpecialCharTok{$}\NormalTok{happiness[df}\SpecialCharTok{$}\NormalTok{happiness }\SpecialCharTok{==} \StringTok{"3"}\NormalTok{] }\OtherTok{\textless{}{-}} \StringTok{"3. Not very happy"}
\NormalTok{df}\SpecialCharTok{$}\NormalTok{happiness[df}\SpecialCharTok{$}\NormalTok{happiness }\SpecialCharTok{==} \StringTok{"4"}\NormalTok{] }\OtherTok{\textless{}{-}} \StringTok{"4. Not at all happy"}

\NormalTok{df}\SpecialCharTok{$}\NormalTok{healthg[df}\SpecialCharTok{$}\NormalTok{healthg }\SpecialCharTok{==} \StringTok{"1"}\NormalTok{] }\OtherTok{\textless{}{-}} \StringTok{"1. Very good"}
\NormalTok{df}\SpecialCharTok{$}\NormalTok{healthg[df}\SpecialCharTok{$}\NormalTok{healthg }\SpecialCharTok{==} \StringTok{"2"}\NormalTok{] }\OtherTok{\textless{}{-}} \StringTok{"2. Good"}
\NormalTok{df}\SpecialCharTok{$}\NormalTok{healthg[df}\SpecialCharTok{$}\NormalTok{healthg }\SpecialCharTok{==} \StringTok{"3"}\NormalTok{] }\OtherTok{\textless{}{-}} \StringTok{"3. Fair"}
\NormalTok{df}\SpecialCharTok{$}\NormalTok{healthg[df}\SpecialCharTok{$}\NormalTok{healthg }\SpecialCharTok{==} \StringTok{"4"}\NormalTok{] }\OtherTok{\textless{}{-}} \StringTok{"4. Poor"}

\NormalTok{df}\SpecialCharTok{$}\NormalTok{happiness }\OtherTok{\textless{}{-}} \FunctionTok{as.factor}\NormalTok{(df}\SpecialCharTok{$}\NormalTok{happiness)}
\NormalTok{df}\SpecialCharTok{$}\NormalTok{healthg }\OtherTok{\textless{}{-}} \FunctionTok{as.factor}\NormalTok{(df}\SpecialCharTok{$}\NormalTok{healthg)}

\FunctionTok{table}\NormalTok{(df}\SpecialCharTok{$}\NormalTok{happiness, df}\SpecialCharTok{$}\NormalTok{healthg)}
\end{Highlighting}
\end{Shaded}

\subsection{Q3. Write code to make this
change.}\label{q3.-write-code-to-make-this-change.}

\begin{Shaded}
\begin{Highlighting}[]
\NormalTok{df}\SpecialCharTok{$}\NormalTok{happiness }\OtherTok{\textless{}{-}} \FunctionTok{as.character}\NormalTok{(df}\SpecialCharTok{$}\NormalTok{happiness)}
\NormalTok{df}\SpecialCharTok{$}\NormalTok{happiness[df}\SpecialCharTok{$}\NormalTok{happiness }\SpecialCharTok{==} \StringTok{"3. Not very happy"} \SpecialCharTok{|}\NormalTok{ df}\SpecialCharTok{$}\NormalTok{happiness }\SpecialCharTok{==} \StringTok{"4. Not at all happy"}\NormalTok{] }\OtherTok{\textless{}{-}} \StringTok{"3. Not very or not at all happy"}
\NormalTok{df}\SpecialCharTok{$}\NormalTok{happiness }\OtherTok{\textless{}{-}} \FunctionTok{as.factor}\NormalTok{(df}\SpecialCharTok{$}\NormalTok{happiness)}
\FunctionTok{table}\NormalTok{(df}\SpecialCharTok{$}\NormalTok{happiness, df}\SpecialCharTok{$}\NormalTok{healthg)}
\end{Highlighting}
\end{Shaded}

\begin{Shaded}
\begin{Highlighting}[]
\FunctionTok{chisq.test}\NormalTok{(df}\SpecialCharTok{$}\NormalTok{happiness, df}\SpecialCharTok{$}\NormalTok{healthg, }\AttributeTok{simulate.p.value =} \ConstantTok{TRUE}\NormalTok{)}\SpecialCharTok{$}\NormalTok{expected}
\CommentTok{\# install.packages(\textquotesingle{}weights\textquotesingle{})}
\FunctionTok{library}\NormalTok{(weights)}
\end{Highlighting}
\end{Shaded}

\begin{verbatim}
## Loading required package: Hmisc
\end{verbatim}

\begin{verbatim}
## 
## Attaching package: 'Hmisc'
\end{verbatim}

\begin{verbatim}
## The following objects are masked from 'package:base':
## 
##     format.pval, units
\end{verbatim}

\begin{Shaded}
\begin{Highlighting}[]
\FunctionTok{wtd.chi.sq}\NormalTok{(df}\SpecialCharTok{$}\NormalTok{happiness, df}\SpecialCharTok{$}\NormalTok{healthg, }\AttributeTok{weight =}\NormalTok{ df}\SpecialCharTok{$}\NormalTok{weight)}
\end{Highlighting}
\end{Shaded}

\subsection{Q4. Provide a short written interpretation of this result
(no more than two
sentences).}\label{q4.-provide-a-short-written-interpretation-of-this-result-no-more-than-two-sentences.}

The results showed p-value is less than 0.0001. This assumes that the
null hypothesis is rejected, which means the two categorical variables
are not independent.

\begin{Shaded}
\begin{Highlighting}[]
\FunctionTok{boxplot}\NormalTok{(df}\SpecialCharTok{$}\NormalTok{age }\SpecialCharTok{\textasciitilde{}}\NormalTok{ df}\SpecialCharTok{$}\NormalTok{healthg)}
\end{Highlighting}
\end{Shaded}

\begin{Shaded}
\begin{Highlighting}[]
\FunctionTok{summary}\NormalTok{(}\FunctionTok{aov}\NormalTok{(df}\SpecialCharTok{$}\NormalTok{age }\SpecialCharTok{\textasciitilde{}}\NormalTok{ df}\SpecialCharTok{$}\NormalTok{healthg))}
\end{Highlighting}
\end{Shaded}

\subsection{Q5. Provide a brief interpretation of this result. One of
the assumptions of ANOVA is that the dependent variable (age in this
case) is normally distributed. Is this a reasonable assumption? Justify
your answer in no more than 2 sentences
total.}\label{q5.-provide-a-brief-interpretation-of-this-result.-one-of-the-assumptions-of-anova-is-that-the-dependent-variable-age-in-this-case-is-normally-distributed.-is-this-a-reasonable-assumption-justify-your-answer-in-no-more-than-2-sentences-total.}

The ANOVA results indicate that there is a significant difference in the
mean \texttt{ages} among the different health levels (\texttt{healthg}),
as the p-value is less than 0.001. This suggests that at least one
\texttt{healthg} group has a mean age that is significantly different
from the others.

Regarding normal distribution, it is reasonable to assume that the age
variable is normally distributed. However, in large samples (e.g., the
residual degrees of freedom here is 1990), ANOVA is still robust due to
the Central Limit Theorem even if the variable might not be normally
distributed.

\begin{Shaded}
\begin{Highlighting}[]
\FunctionTok{shapiro.test}\NormalTok{(df}\SpecialCharTok{$}\NormalTok{age)  }\CommentTok{\# test if df$age is normally distributed {-}{-}\textgreater{} \textquotesingle{}p \textless{} 0.001\textquotesingle{} means it\textquotesingle{}s not}
\end{Highlighting}
\end{Shaded}

\subsection{Q6. Find some data on your own. Ensure that the data has one
categorical variable with at least three levels, and one continuous
numeric variable. Your null hypothesis is that the numeric variable does
not vary across the groups. Use one-way ANOVA or Kruskal-Wallis to
analyse these data. In your answer, provide 1) a link to the data 2) a
description of the data, 3) the results of your analysis and 4) an
interpretation of your results. Make it look nice and pretty, and ensure
it is no more than 1 page in
length.}\label{q6.-find-some-data-on-your-own.-ensure-that-the-data-has-one-categorical-variable-with-at-least-three-levels-and-one-continuous-numeric-variable.-your-null-hypothesis-is-that-the-numeric-variable-does-not-vary-across-the-groups.-use-one-way-anova-or-kruskal-wallis-to-analyse-these-data.-in-your-answer-provide-1-a-link-to-the-data-2-a-description-of-the-data-3-the-results-of-your-analysis-and-4-an-interpretation-of-your-results.-make-it-look-nice-and-pretty-and-ensure-it-is-no-more-than-1-page-in-length.}

\begin{Shaded}
\begin{Highlighting}[]
\FunctionTok{library}\NormalTok{(here)}
\FunctionTok{here}\NormalTok{()}
\NormalTok{pitchfork\_albums }\OtherTok{\textless{}{-}} \FunctionTok{data.frame}\NormalTok{(}\FunctionTok{read.csv}\NormalTok{(}\FunctionTok{paste0}\NormalTok{(}\FunctionTok{here}\NormalTok{(), }\StringTok{"/Assignments/Assignment3/data/pitchfork\_reviews.csv"}\NormalTok{)))}

\NormalTok{pitchfork\_albums\_updated }\OtherTok{\textless{}{-}}\NormalTok{ pitchfork\_albums[pitchfork\_albums}\SpecialCharTok{$}\NormalTok{score }\SpecialCharTok{!=} \StringTok{"Not Available"} \SpecialCharTok{\&}
    \SpecialCharTok{!}\NormalTok{(pitchfork\_albums}\SpecialCharTok{$}\NormalTok{year }\SpecialCharTok{==} \StringTok{"Not Available"}\NormalTok{), ]}
\NormalTok{pitchfork\_albums\_updated}\SpecialCharTok{$}\NormalTok{cnt }\OtherTok{\textless{}{-}} \DecValTok{1}
\NormalTok{pitchfork\_albums\_updated}\SpecialCharTok{$}\NormalTok{score }\OtherTok{\textless{}{-}} \FunctionTok{as.numeric}\NormalTok{(pitchfork\_albums\_updated}\SpecialCharTok{$}\NormalTok{score)}
\NormalTok{pitchfork\_albums\_updated}\SpecialCharTok{$}\NormalTok{year }\OtherTok{\textless{}{-}} \FunctionTok{as.numeric}\NormalTok{(pitchfork\_albums\_updated}\SpecialCharTok{$}\NormalTok{year)}

\FunctionTok{boxplot}\NormalTok{(pitchfork\_albums\_updated}\SpecialCharTok{$}\NormalTok{score }\SpecialCharTok{\textasciitilde{}}\NormalTok{ pitchfork\_albums\_updated}\SpecialCharTok{$}\NormalTok{year)}
\end{Highlighting}
\end{Shaded}

\begin{Shaded}
\begin{Highlighting}[]
\NormalTok{agg\_albums1 }\OtherTok{\textless{}{-}} \FunctionTok{aggregate}\NormalTok{(pitchfork\_albums\_updated}\SpecialCharTok{$}\NormalTok{cnt, }\AttributeTok{by =} \FunctionTok{list}\NormalTok{(pitchfork\_albums\_updated}\SpecialCharTok{$}\NormalTok{year),}
    \AttributeTok{FUN =}\NormalTok{ sum)}

\NormalTok{pitchfork\_albums\_updated}\SpecialCharTok{$}\NormalTok{year[pitchfork\_albums\_updated}\SpecialCharTok{$}\NormalTok{year }\SpecialCharTok{\textless{}=} \DecValTok{1995}\NormalTok{] }\OtherTok{\textless{}{-}} \DecValTok{1}
\NormalTok{pitchfork\_albums\_updated}\SpecialCharTok{$}\NormalTok{year[pitchfork\_albums\_updated}\SpecialCharTok{$}\NormalTok{year }\SpecialCharTok{\textgreater{}=} \DecValTok{1996} \SpecialCharTok{\&}\NormalTok{ pitchfork\_albums\_updated}\SpecialCharTok{$}\NormalTok{year }\SpecialCharTok{\textless{}=}
    \DecValTok{2003}\NormalTok{] }\OtherTok{\textless{}{-}} \DecValTok{2}
\NormalTok{pitchfork\_albums\_updated}\SpecialCharTok{$}\NormalTok{year[pitchfork\_albums\_updated}\SpecialCharTok{$}\NormalTok{year }\SpecialCharTok{\textgreater{}} \DecValTok{2003} \SpecialCharTok{\&}\NormalTok{ pitchfork\_albums\_updated}\SpecialCharTok{$}\NormalTok{year }\SpecialCharTok{\textless{}=}
    \DecValTok{2008}\NormalTok{] }\OtherTok{\textless{}{-}} \DecValTok{3}
\NormalTok{pitchfork\_albums\_updated}\SpecialCharTok{$}\NormalTok{year[pitchfork\_albums\_updated}\SpecialCharTok{$}\NormalTok{year }\SpecialCharTok{\textgreater{}} \DecValTok{2008} \SpecialCharTok{\&}\NormalTok{ pitchfork\_albums\_updated}\SpecialCharTok{$}\NormalTok{year }\SpecialCharTok{\textless{}=}
    \DecValTok{2013}\NormalTok{] }\OtherTok{\textless{}{-}} \DecValTok{4}
\NormalTok{pitchfork\_albums\_updated}\SpecialCharTok{$}\NormalTok{year[pitchfork\_albums\_updated}\SpecialCharTok{$}\NormalTok{year }\SpecialCharTok{\textgreater{}} \DecValTok{2013} \SpecialCharTok{\&}\NormalTok{ pitchfork\_albums\_updated}\SpecialCharTok{$}\NormalTok{year }\SpecialCharTok{\textless{}=}
    \DecValTok{2018}\NormalTok{] }\OtherTok{\textless{}{-}} \DecValTok{5}
\NormalTok{pitchfork\_albums\_updated}\SpecialCharTok{$}\NormalTok{year[pitchfork\_albums\_updated}\SpecialCharTok{$}\NormalTok{year }\SpecialCharTok{\textgreater{}} \DecValTok{2018}\NormalTok{] }\OtherTok{\textless{}{-}} \DecValTok{6}

\NormalTok{agg\_albums }\OtherTok{\textless{}{-}} \FunctionTok{aggregate}\NormalTok{(pitchfork\_albums\_updated}\SpecialCharTok{$}\NormalTok{score, }\AttributeTok{by =} \FunctionTok{list}\NormalTok{(pitchfork\_albums\_updated}\SpecialCharTok{$}\NormalTok{year),}
    \AttributeTok{FUN =}\NormalTok{ mean)}
\FunctionTok{names}\NormalTok{(agg\_albums) }\OtherTok{\textless{}{-}} \FunctionTok{c}\NormalTok{(}\StringTok{"Year"}\NormalTok{, }\StringTok{"ave\_scores"}\NormalTok{)}
\FunctionTok{plot}\NormalTok{(agg\_albums}\SpecialCharTok{$}\NormalTok{Year, agg\_albums}\SpecialCharTok{$}\NormalTok{ave\_scores)}
\end{Highlighting}
\end{Shaded}

\begin{Shaded}
\begin{Highlighting}[]
\NormalTok{agg\_albums2 }\OtherTok{\textless{}{-}} \FunctionTok{aggregate}\NormalTok{(pitchfork\_albums\_updated}\SpecialCharTok{$}\NormalTok{cnt, }\AttributeTok{by =} \FunctionTok{list}\NormalTok{(pitchfork\_albums\_updated}\SpecialCharTok{$}\NormalTok{year),}
    \AttributeTok{FUN =}\NormalTok{ sum)}

\FunctionTok{boxplot}\NormalTok{(pitchfork\_albums\_updated}\SpecialCharTok{$}\NormalTok{score }\SpecialCharTok{\textasciitilde{}}\NormalTok{ pitchfork\_albums\_updated}\SpecialCharTok{$}\NormalTok{year)}
\end{Highlighting}
\end{Shaded}

\begin{Shaded}
\begin{Highlighting}[]
\FunctionTok{summary}\NormalTok{(}\FunctionTok{aov}\NormalTok{(pitchfork\_albums\_updated}\SpecialCharTok{$}\NormalTok{score }\SpecialCharTok{\textasciitilde{}}\NormalTok{ pitchfork\_albums\_updated}\SpecialCharTok{$}\NormalTok{year))}
\end{Highlighting}
\end{Shaded}

\subsubsection{Introduction}\label{introduction}

Pitchfork is an American online music publication founded in 1996 by
Ryan Schreiber in Minneapolis. Since then, it began to gain popularity
among indie music fans and till now it has become a professionalmusic
publication loved by a lot of people.

In this project I used
\href{https://www.kaggle.com/datasets/timstafford/pitchfork-reviews}{``Pitchfork
Reviews: Music Critiques Over the Years''} Kaggle dataset uploaded by
Tim Stafford.

\subsubsection{Data Preparation and Descriptive
Analysics}\label{data-preparation-and-descriptive-analysics}

This dataset has 25708 rated album records that were released from 1952
to 2023. First, I screened out 24699 album records that do not have
missing information about scores or years. Then I aggregated the album
numbers by their released years and found that before 2002, the number
of albums rated each year increased with years. And after 2002, the
albums rated each year range stably from 850 to 1250 (which means
averagely they publish three to four reviews every day). I chose 1996
and 2002 as two important years and since recategorize the continuous
\texttt{year} variable into a 1-6 categorical variable, which are
\textless= 1995, 1996-2003, 2004-2008, 2009-2013, 2014-2018, 2019-2023).

\subsubsection{ANOVA Aanalysis}\label{anova-aanalysis}

I did a ANOVA test with the continuous \texttt{score} variable and the
categorical \texttt{year} variable. The result shows that the
\texttt{p-value} is less than 0.001, which means there exist significant
score differences between groups. According to the box plot, it is
obviously that the scores in the ``before 1996'' group are significantly
higher than the other gourps, with an average score of 8.7/10.0. This is
partly because all the albums released before 1996 that Pitchfork rated
were not contemporary albums. They either use their ``Sunday Review''
section to review the albums from the past once a week, or they would
review those legacy albums after one artist make anniversity reissues
(e.g., Aphex Twin, Joni Mitchell) or deceased (e.g., David Bowie,
Prince). These ``old'' albums are typically rated higher.

The other interesting phenomenon is the scores given were gradually
increasing since 1996. And along with it, the variances between album
scores in a single year are decreasing. This would mean that Pitchfork
does not give that many incredibly ``lower'' scores (\textless{}
4.0/10.0), neither does it give those ``super high'' scores that often
(\textgreater{} 9.5/10.0). This is how Pitchfork has been evolving, that
it is not that bold and brashy as it used me, instead, it became more
``conservative'' and ``safe'' in recent years.

\end{document}
